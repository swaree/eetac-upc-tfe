\begin{landscape}
\chapter{Sustainability and ethical implications analysis}

The current UPC thesis guidelines enforce that all Bachelor's and Master's theses shall include a section where the environmental and social impact of the thesis is estimated. More information can be found \href{https://govern.upc.edu/ca/consell-de-govern/consell-de-govern/sessio-07-2023-del-consell-de-govern/comissio-docencia-i-politica-academica-pendent-celebracio/informacio-de-la-guia-per-incorporar-lanalisi-de-sostenibilitat-i-implicacions-etiques-en-els-tfe/informacio-de-la-guia-per-incorporar-lanalisi-de-sostenibilitat-i-implicacions-etiques-en-els-tfe/@@display-file/visiblefile/Guia%20Informe%20Sostenibilitat%20TFG_TFM_versi%C3%B3%20UPC.pdf}{here} (in Catalan). You can follow the ``matrix" layout or write it out.

\hspace*{\fill}\vspace*{\fill}

%\begin{landscape}
    \begin{table}[htbp!]
        \centering
        \footnotesize
        \renewcommand{\arraystretch}{1.07}
        \begin{tabular}{@{}lm{6.5cm}m{7cm}m{5.5cm}@{}}
            \toprule
             & \bfseries Project development & \bfseries Exploitation & \bfseries Risks and limitations\\
            \midrule
            \bfseries Environmental & Quantify the environmental impact of the project. What measures have you taken to reduce the impact? Have you quantified this reduction? Does your design follow the cradle-to-cradle philosophy? · What is the origin of the raw materials and/or materials used? Do your suppliers publish environmental reports? · Do your suppliers follow the RoHS directive? Do your suppliers follow the RBA Code of Conduct? & What resources do you estimate will be used during the project's lifetime? What will be the environmental impact of these resources? · Will the project reduce the use of other resources? Overall, will the use of the project improve or worsen the ecological footprint? · When the life of the project comes to an end, what waste is generated? How the environmental impact of dismantling can be reduced? · Could the project be carried out with less environmental impact? & Could any scenarios that might increase the footprint of the project arise? · If you did the project again, could it be done with fewer resources? Can it be designed again with reused materials? · What have been the main limitations of the environmental analysis of your proposal?\\
            \midrule
            \bfseries Economic & Quantify the project's cost (human and material resources). What decisions have you taken to reduce the cost? Have you quantified the savings? · Is the estimated cost similar to the final cost? Justify the differences (lessons learned). & What is the estimated cost of the project over its lifetime? Could this cost be reduced to make the project more feasible? · Have you considered the cost of adjustments, updates, or repairs over the life of the project? · Would the dismantling of the project incur any additional costs? · Could any other project benefit from the results of this one? & Could any scenarios arise that may jeopardize the viability of the project? · What have been the main limitations of the economic analysis of your proposal?\\
            \midrule
            \bfseries Social & Does this project involve significant reflections on the personal, professional, or ethical standards of the people working on the project? Has inclusive and non-sexist language been used? · What is the sector's current situation related to the project? · Do the distributors, manufacturers, suppliers, and retailers meet public ethical or conduct codes? & Who benefits from the use of the project? Is there any group that may be adversely affected by the project? If so, to what extent? · To what extent does the project solve the problem initially raised? · Are there other ways of implementing the project that lead to different social impacts? · Does the project avoid biases, stereotypes, and gender roles? · Have you considered the usability of your product for people with diverse needs (age, gender, sex, functional diversity, cultural diversity, etc.)? Are there barriers to using it? & Could any scenarios arise to make the project detrimental to any particular segment of the population? · Could the project create any dependency that might leave users in a weak position? · What have been the main limitations of the social analysis of your proposal?\\
            \bottomrule
        \end{tabular}
        \caption{Sustainability matrix.}
        \label{tab:seia}
    \end{table}
\end{landscape}